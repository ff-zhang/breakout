\documentclass{article}

%% Page Margins %%
\usepackage{geometry}
\geometry{
    top = 0.75in,
    bottom = 0.75in,
    right = 0.75in,
    left = 0.75in,
}

\usepackage{amsmath}
\usepackage{graphicx}
\usepackage{parskip}

\title{Project Report: Milestone 1}

\author{Janssen Myer Rambaud (1008107004), Felix Zhang (1007650212)}

\begin{document}
\maketitle

\section{Part I: Planning and Configuration}

\begin{enumerate}
\item Breakout Plan:

\item Translate your plan into the .data section of your breakout.asm program. Assemble your program in MARS and inspect memory to ensure it matches your plan. Include a screenshot (or multiple screenshots) of memory demonstrating that it has been laid out according to your plan.

\begin{figure}[ht!]
    \centering
    % \includegraphics[width=0.8\textwidth]{milestone1_memory.png}
    \caption{Screenshot of memory.}
    \label{f:part1_memory_writer}
\end{figure}

\newpage

\section{Part II: Milestone 1}

\item Draw the scene (Milestone 1)

\begin{figure}[ht!]
    \centering
    \includegraphics[width=0.8\textwidth]{milestone1_drawing.png}
    \caption{The static scene of Milestone 1 Drawing.}
    \label{f:milestone1_drawing}
\end{figure}

\end{enumerate}


\newpage

\section{Part III: Milestone 2}

\item \textbf{QUESTION: } How will the ball change directions when it collides? \\[0.4em]
The rules governing how the ball changes directions when it collides are as follows. \\[0.4em]
If it collided with the paddle, then the resulting direction depends on which section of the paddle it hit.
Hitting the leftmost third results it travelling north-east (at a $45^\circ$ angle), the middle third results it travelling north, and the rightmost third results in it travelling north-west (at a $45^\circ$ angle). \\[0.4em]
Otherwise, if it collided with a non-paddle object then the resulting direction depends on the original direction.
If the original direction of ball forms the angle $\theta$ with the surface it collided with, 
then the resulting direction is the reflection of the original direction across the line perpendicular to the collision surface away from the collision point.
For example, if the ball is travelling north-west (at a $45^\circ$ angle) and collides with the left wall, then it begins travelling in the north-east direction (at a $45^\circ$ angle).
Note this also means if the ball was travelling perpendicular to surface with which it collided, then after the collision, it is moving in the reverse direction.
So, if the ball was travelling north before colliding with the bottom of a brick, then it is travelling south after the collision.
\item Upload breakout.asm to MarkUs so that you have a snapshot of your progress so far.

\section{Part IV: Milestone 3}
\item Upload breakout.asm to MarkUs so that you have a snapshot of your progress so far.
\end{document}